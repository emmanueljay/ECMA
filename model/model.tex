%%%%%%%%%%%%%%%%%%%%%%%%%%%%%%%%%%%%%%%%%
% Thin Sectioned Essay
% LaTeX Template
% Version 1.0 (3/8/13)
%
% This template has been downloaded from:
% http://www.LaTeXTemplates.com
%
% Original Author:
% Nicolas Diaz (nsdiaz@uc.cl) with extensive modifications by:
% Vel (vel@latextemplates.com)
%
% License:
% CC BY-NC-SA 3.0 (http://creativecommons.org/licenses/by-nc-sa/3.0/)
%
%%%%%%%%%%%%%%%%%%%%%%%%%%%%%%%%%%%%%%%%%

%----------------------------------------------------------------------------------------
% PACKAGES AND OTHER DOCUMENT CONFIGURATIONS
%----------------------------------------------------------------------------------------

\documentclass[a4paper, 11pt]{article} % Font size (can be 10pt, 11pt or 12pt) and paper size (remove a4paper for US letter paper)

\usepackage[protrusion=true,expansion=true]{microtype} % Better typography
\usepackage[utf8]{inputenc}  
\usepackage{graphicx} % Required for including pictures
\usepackage{wrapfig} % Allows in-line images
\usepackage[a4paper, margin={3cm, 3cm}]{geometry}

\usepackage{mathenv}
\usepackage{amsmath,amsfonts,amssymb}
\usepackage{setspace}
\usepackage{bbm}
\usepackage{bm}
\usepackage{layout}

\usepackage{mathpazo} % Use the Palatino font
\usepackage[T1]{fontenc} % Required for accented characters
\linespread{1.05} % Change line spacing here, Palatino benefits from a slight increase by default

\usepackage{listings} % Pour ajouter du code
\usepackage{xcolor}
\lstset { %
    language=C++,
    backgroundcolor=\color{black!5}, % set backgroundcolor
    basicstyle=\footnotesize,% basic font setting
}

\makeatletter
\renewcommand\@biblabel[1]{\textbf{#1.}} % Change the square brackets for each bibliography item from '[1]' to '1.'
\renewcommand{\@listI}{\itemsep=0pt} % Reduce the space between items in the itemize and enumerate environments and the bibliography

\renewcommand{\maketitle}{ % Customize the title - do not edit title and author name here, see the TITLE block below
\begin{flushright} % Right align
{\LARGE\@title} % Increase the font size of the title

\vspace{30pt} % Some vertical space between the title and author name

{\large\@author} % Author name
\\\@date % Date

\vspace{20pt} % Some vertical space between the author block and abstract
\end{flushright}
}

%----------------------------------------------------------------------------------------
% TITLE
%----------------------------------------------------------------------------------------

\title{\textbf{Modélisation projet ECMA}} % Subtitle

\author{\textsc{Dufour Maxime, Jay Emmanuel} % Author
\\{\textit{ENSTA ParisTech - CNAM}}} % Institution

\date{\today} % Date

%----------------------------------------------------------------------------------------

\begin{document}

\vspace{200pt}

\maketitle % Print the title section

\section{Exercice 1 : Modélisation du problème en ignorant la connexité}

\subsection{Q1 : Modélisation du problème}

\paragraph{Variables :}
Les variables de ce problème sont les $x_{i,j}$ qui sont des variables binaires qui valent 1 si la maille $(i,j)$ est retenue

\paragraph{Fonction objectif :}
Nous cherchons à maximiser la surface retenue donc :
$$
  \max \sum_{(i,j)\in M} x_{i,j}
$$

\paragraph{Contrainte :}
Sous la contrainte \textit{fractionnaire} :
\begin{equation}  
  \frac{\sum_{(i,j)\in M} H_{i,j}^p C_{i,j}^p x_{i,j}}{\sum_{(i,j)\in M} C_{i,j}^p x_{i,j}}
  + \frac{\sum_{(i,j)\in M} H_{i,j}^a C_{i,j}^a x_{i,j}}{\sum_{(i,j)\in M} C_{i,j}^a x_{i,j}} \geq 2
\label{c1}
\end{equation}  

\subsection{Q2 : Linéarisation de la contrainte fractionnaire}
\paragraph*{}
L'idée est de séparer la contrainte fractionnaire en deux parties que l'on va poser égales à des variables réelles (contraintes quadratiques) : 
\begin{equation}  
  \frac{\sum_{(i,j)\in M} H_{i,j}^p C_{i,j}^p x_{i,j}}{\sum_{(i,j)\in M} C_{i,j}^p x_{i,j}} = \alpha^p
\label{alphapq}
\end{equation}
et
\begin{equation}  
  \frac{\sum_{(i,j)\in M} H_{i,j}^a C_{i,j}^a x_{i,j}}{\sum_{(i,j)\in M} C_{i,j}^a x_{i,j}} = \alpha^a
\label{alphaaq}
\end{equation} 
ce qui nous permet d'écrire la contrainte fractionnaire \eqref{c1} comme cela :
\begin{equation}  
  \alpha^p + \alpha^a \geq 2
\label{c1l}
\end{equation}  

\paragraph*{}
Nous devons donc maintenant linéariser \eqref{alphapq} et \eqref{alphaaq}. On utilise pour cela la linéarisation simple entre un booléen et une variable entière. Soit $k \in {a,p}$. On peut écrire les équations \eqref{alphapq} et \eqref{alphaaq}, en multipliant par le dénominateur, comme : 
$$
  \sum_{(i,j)\in M} H_{i,j}^k C_{i,j}^k x_{i,j} = \alpha^k * \sum_{(i,j)\in M} C_{i,j}^k x_{i,j}
$$
et donc 
$$
  \sum_{(i,j)\in M} H_{i,j}^k C_{i,j}^k x_{i,j} = \sum_{(i,j)\in M} C_{i,j}^k \alpha^k * x_{i,j} 
$$
Et le travail revient donc à linéariser les couples de variables $ \alpha^k * x_{i,j} $. Pour cela, nous introduisons un nouveau jeu de variables $y_{i,j}^k$ pour $k \in {a,p}$ et $(i,j)\in M$ qui vont représenter $ \alpha^k * x_{i,j} $.

\paragraph*{}
Soit $k \in {a,p}$ et $(i,j)\in M$, alors on ajoute les contraintes suivantes :
$$ 
  y_{i,j}^k \leq u^k * x_{i,j} 
$$
$$
  y_{i,j}^k \leq \alpha^k 
$$
$$
  y_{i,j}^k \geq (x_{i,j}-1)*u^k + \alpha^k 
$$
$$
  y_{i,j}^k \geq 0 
$$
Avec $u^k$ une borne supérieure que l'on définit dans les données.

\paragraph*{}
La contrainte fractionnaire \eqref{c1} se linéarise donc, en ajoutant les variables $\alpha^k$ et $y_{i,j}^k$ en : 
$$
  \sum_{(i,j)\in M} H_{i,j}^k C_{i,j}^k x_{i,j} = \sum_{(i,j)\in M} C_{i,j}^k y_{i,j}^k 
$$
$$ 
  y_{i,j}^k \leq u^k * x_{i,j} 
$$
$$
  y_{i,j}^k \leq \alpha^k 
$$
$$
  y_{i,j}^k \geq (x_{i,j}-1)*u^k + \alpha^k 
$$
$$
  y_{i,j}^k \geq 0 
$$
Avec la contrainte supplémentaire \eqref{c1l} :
$$ 
  \alpha^p + \alpha^a \geq 2
$$

\section{Exercice 2 : Modélisation de la connexité}
\paragraph*{}
Pour modéliser cette contrainte de connexité, nous avons besoin d'introduire des contraintes qui vont indiquer une distance à une origine que le programme déterminera. Nous allons donc introduire des variables booléennes $s_{i,j}^k$ avec $k$ étant la ``distance'' à l'origine. Nous devons donc introduire plusieurs contraintes :

\paragraph*{Sélection de l'origine.}
Nous devons avoir une seule variable $s_{i,j}^0$ qui correspond à l'origine :
$$ 
  \sum_{(i,j)\in M} s_{i,j}^0 = 1
$$

\paragraph*{Chaque maille retenue doit avoir une distance à l'origine.} Cette contrainte correspond à la contrainte de connexité
$$ 
  \sum_{k = 0}^{k_{max}} s_{i,j}^k = x_{i,j}  \qquad \forall (i,j) \in M
$$ 

\paragraph*{La distance d'une maille est conditionnée par celle de ses voisins.} Pour avoir une distance k, nous avons besoin qu'un des voisins de la maille ait une distance $k-1$
$$ 
  s_{i,j}^k \leq s_{i-1,j}^{k-1} + s_{i,j-1}^{k-1} + s_{i+1,j}^{k-1} + s_{i,j+1}^{k-1} \qquad \forall k \in [|0,k_{max}|]
$$ 

\paragraph*{$k_{max}$ :} Nous avons à déterminer la valeur maximum que ce k peut prendre. Nous pouvons considérer $n*m$ en premier abord.



\end{document}

